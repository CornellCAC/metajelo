We target a specific but very common use case. In most applied sciences,
it has become common publication practice to provide evidence of the
statistical or laboratory data underlying the conclusions. This is done
to support reproducibility and replicability of the scientific
findings.%
\footnote{There is considerable heterogeneity in the use of the terms
	``reproducibility'' and ``replicability''. In this paper, we will adopt
	the following definitions: reproducibility is ``the ability of a
	researcher to duplicate the results of a prior study using the same
	materials and procedures as were used by the original investigator,''
	\parencite{BollenSocialBehavioralEconomic2015} whereas replicability
	differs in that ``new data are collected.'' (ibidem).} 
Journals with a
data deposit policy have stored the supplementary materials on journal
websites, often as simple web-based ZIP archives. While ensuring that
the materials are preserved as long as the journal is active
(permanence) and are accessible to any reader of the original article
(subject to pay walls) (accessibility), certain shortcomings became
apparent. Very large datasets and datasets with confidentiality concerns
were nearly always out of scope. More recently, journals have leveraged
either dedicated, journal-branded views onto larger archives (e.g,
Dataverse, Figshare), built their own data archive infrastructure
(\urlcite{https://www.elsevier.com/authors/author-services/research-data}{Elsevier/Mendeley}), or have allowed for data and code to be stored
more generally on any of a curated list of ``trusted'' or ``approved''
whitelist of third-party repositories.%
\footnote{\url{https://f1000research.com/for-authors/data-guidelines}, \url{https://www.nature.com/sdata/policies/repositories}}
Each of these alternatives rely
on a journal or publisher ``vetting'' the repositories and ascertaining
that it meets some set of criteria.  While some third-party vetting of
repositories exists,%
\footnote{CoreTrustSeal, \url{https://www.coretrustseal.org/}}
it is far from being universally accepted at this
time. In all cases known to us, the support for restricted-access
repositories is quite limited. Thus, most of the known support for
third-party repositories does not provide much information about
accessibility (the presumption is that access is open), nor about the
permanence of the repositories - this is presumably one of the
evaluation criteria that journals and publishers use, but is not clearly
defined as such. In fact, at least one of the consulted publishers
explicitly allows for quite transitory repositories for code, without
clearly distinguishing that from archives that are more permanent.%
\footnote{F1000Research (\url{https://f1000research.com/for-authors/data-guidelines}) allows for code deposits through github.com, which has no mandate to preserve, and allows code owners to delete materials at any time without restrictions.}
Nevertheless, much of the information about persistence of archives and
materials stored within those archives is available, albeit in
idiosyncratic and non-machine readable form. Consider only the case of
national archives (e.g., the \urlcite{https://www.archives.gov/dc/researcher-info}{U.S. National Archives} or the \urlcite{http://www.archives-nationales.culture.gouv.fr/}{\textit{Archives Nationales} in France}). 
In general, data stored in national archives is
permanently archived; if it is not, this is clearly documented.%
\footnote{For instance, the program code for the Business Register is destroyed when a new system is put in place - they are never kept \parencite{U.S.CensusBureauRecordsControlSchedule2009}. Unedited master files for the American Community Survey are destroyed 6 years after the Edited master files are verified, unless still needed ``for Census operations'' \parencite{U.S.CensusBureauRecordsControlSchedule1999}.}
Furthermore, access is generally not restricted - if it is, this is
clearly documented. However, materials in national archives do have
certain restrictions - they may require sending in a written request, or
a physical visit to a location with copies of the data. Thus, while the
information may satisfy the publication requirements of even the most
open journal, there is no robust and standardized way of documenting the
additional restrictions on access that persist. In proposing the
metadata package outlined in this article, we attempt to improve on this
situation. By providing a sparse but sufficient encapsulation of the
information collected from authors, archives, and other third-parties,
we create greater transparency about the data supporting the research.
By relying on existing metadata schemas and metadata content, we
minimize the effort by all parties involved, increasing the likelihood
of adoption. And by intrinsically addressing the possibility that the
information obtained at the time of publication may differ from later
requests for the same information, we provide the tools to journals,
publishers, and their editors to document that the decision to publish
was based on adequate information at the time of the publication (or
acceptance decision).

\subsection{Why we cannot use the current infrastructure}
Examples here: see the supplementary materials: license information incomplete (DataCite), retention policy incomplete (re3data), links incomplete (Scholix). These may all become complete, but currently are not. Journals are implementing their replication policies now, and need the information now.