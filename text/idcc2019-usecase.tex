We target a specific but very common use case. A researcher has written a paper with empirical content, and is required by the journal's data and code availability policy to prepare a ``replication package.'' The journal's policy requires that the code and data be accessible to others, but does not require deposit of the materials as a ``supplementary file,'' i.e., as a ZIP file on their website. However, in all cases, the journal wishes to ascertain three key attributes of the replication package or packages:
\begin{itemize}
    \item the existence of the package
    \item the access rules to the package (license, terms of use)
    \item the persistence of the package
\end{itemize}
In an ideal \ac{FAIR} and open-data compatible scenario, the existence of the package can be ascertained in a reputable repository, it is made available under an open license (e.g., CC-BY), and it is available ``forever''. These attributes, however, need to be discovered, and this needs to happen in a scalable, thus automated fashion, as it should be feasible to do so for all articles, submitted to any journal. 






\subsection{Common Denominator}
We note that we chose plausible cases where the metadata crawl did not yield satisfactory information. For both data repositories, examples could have been chosen where the metadata might have yielded more satisfactory results. However, the methods of collecting the information must accommodate all scenarios, including the ones where the automated mechanisms fail.

We accomplish this by designing a derivative metadata package that can be populated at the point of first use: the journal submission system, or if the researcher uses a reproducible workflow, at data acquisition by the researcher. An associated light query system can first exhaust all metadata crawls, and pre-fill any fields. However, when ambiguous responses are obtained (as in Use Case 1), or no information is available (as in both use cases), the researcher can provide guided or verbatim answers. At both points in time, the researcher has the best incentives to provide the information accurately -- the acceptance of the submission may depend on the accuracy of the information -- and the most timely recollection of where to obtain the information.


\subsection{Why we cannot use the current infrastructure}
Examples here: see the supplementary materials: license information incomplete (DataCite), retention policy incomplete (re3data), links incomplete (Scholix). These may all become complete, but currently are not. Journals are implementing their replication policies now, and need the information now.