Reproducibility and replicability of scientific findings has been given great scrutiny in recent years \parencite{CamererEvaluatingreplicabilitylaboratory2016,Collaboration2015-ev,Klein2014,FanelliOpinionsciencereally2018}.
%
Actual published individual reproductions or replications are traditionally not very common \parencite[in economics, see][]{BellMiller2013b,Duvendack2017}. In part, this is because it often was difficult to find the materials required to conduct reproducibility or replication exercises \parencite{Dewald1986,McCullough2006,McCullough03}.  

Scientific journals, whether run by publishing companies (Springer, Elsevier, etc.) or learned societies (American Economic Association, Midwest Political Science Association, American Statistical Association, Royal Statistical Society, to name just a few in the social and statistical sciences), have been playing an important role in supporting these efforts for many years \parencite{stodden_enhancing_2016}, and continue to explore novel and better ways of doing so. More and more journals are adopting ``data and code availability'' policies, though some doubt has been cast on their effectiveness \parencite{stodden_toward_2013,Stoddenempiricalanalysisjournal2018,Hoeffler2017}. Some of the lack of replicability identified by recent studies \parencite{Hoeffler2017a,Chang2017,ChangLi2015,CamererEvaluatingreplicabilitylaboratory2016,Stoddenempiricalanalysisjournal2018}  is despite the fact that journals have these policies. One issue is the lack of consistent, reliable metadata on the materials provided to journals, and in particular those provided through third-party locations.

Several journals have been hosting ``supplementary materials'' on their own journal websites or on affiliated repositories (e.g., Harvard Dataverse, Figshare) in support of reproducibility of the work described in published scientific articles. Data and code deposits are requested when authors' work has been (conditionally) accepted after peer review, or, less frequently, as part of the original manuscript submission process. In doing so, they assume for themselves (or delegate to a single trusted third party) the curation role for these materials, and can therefore know with certainty how long and how accessible these materials are to be preserved.

Authors are increasingly being encouraged and trained in reproducible methods from the outset of their research projects, rather than performing ex-post documentation. This includes carefully documenting provenance of third-party datasets being used, and properly curating generated datasets (surveys, collected data, etc.) in data archives as soon as possible. Furthermore, in at least some social sciences, the use of pre-existing but non-public data has increased substantially.  Confidentiality and licensing constraints prevent authors from depositing such data in open archives. Journals must rely on an increasingly diverse cadre of data-holding institutions, not all of which are ``archives'' in the traditional sense, while satisfying increasing scrutiny of the provenance of the research results published by them. Both scenarios - early and third-party deposit of data and use of restricted-access data - make it difficult for journals and traditional archives to carry out their curation role. The resulting lack of transparency in data provenance is detrimental to the overall effort of increasing transparency in the sciences.

The approach outlined in this article proposes a metadata package, derived from existing metadata schemata where possible, that provides a lightweight approach to ameliorating this problem. In particular, the proposed metadata package, called \metajelo (\underline{meta}data package for \underline{j}ournals to support \underline{e}xternal \underline{l}inked \underline{o}bjects) documents the key characteristics that journals care about in the case of supplementary materials that are held by third parties: existence, accessibility, and permanence. Our intent in defining the metadata package is two-fold. First, the package enables  authors to provide the information as they submit articles to journals, allowing informed editorial decisions to be made. Second, at the time of publication, the information is made public, providing robust documentation on data provenance in an immutable package, in a compact fashion.  The package allows for better documentation of any data, regardless of the difficulty of access.   Thus the information provided for less accessible (non-public data) is improved by treating it symmetrically with open access data, therefore increasing the transparency of what up until now has been very opaque.

We start by providing some background. We describe the use case motivating our approach, with detailed use cases provided in the appendix. We relate our approach to existing metadata, both in terms of structure as of content, and then describe the metadata package. We conclude by discussing some usability issues for three contributors or consumers of this information, and an outlook on a possible implementation.