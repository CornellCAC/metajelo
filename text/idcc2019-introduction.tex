Reproducibility and replicability of scientific findings has been given greater scrutiny in recent years (REFS). Scientific journals, whether run by publishing companies (Springer, Elsevier, etc.) or learned societies (American Economic Association, Midwest Political Science Association, American Statistical Association, Royal Statistical Society, to name just a few in the social and statistical sciences), have been playing an important role in supporting these efforts for many years, and continue to explore novel and better ways of doing so. In particular, several journals have been hosting ``supplementary materials'' on their own journal websites or on affiliated repositories (e.g., Harvard Dataverse, Figshare) in support of reproducibility of the work described in published scientific articles. Data and code deposits are requested after authors' work has been (conditionally) accepted after peer review, or, less frequently, as part of the original manuscript submission process. In doing so, they assume for themselves (or delegate to a single trusted third party) the curation role for these materials, and can therefore know with certainty how long and how accessible these materials are to be preserved.

Authors are increasingly being encouraged and trained in reproducible methods from the outset of their research projects, rather than performing ex-post documentation. This includes carefully documenting provenance of third-party datasets being used, and properly curating generated datasets (surveys, collected data, etc.) in data archives as soon as possible. Furthermore, in at least some social sciences, the use of pre-existing but non-public data has increased substantially.  Confidentiality and licensing constraints prevent authors from depositing such data in open archives. Both scenarios - early deposit and use of restricted-access data - make it difficult for journals and traditional archives to carry out their curation role. Journals must rely on an increasingly diverse cadre of data-holding institutions, not all of which are ``archives'' in the traditional sense, while satisfying increasing scrutiny of the provenance of the research results published by them. 

The approach outlined in this article proposes a metadata package, derived from existing metadata where possible, that provides a lightweight approach to ameliorating this problem. In particular, the proposed metadata package documents the key characteristics that journals care about in the case of supplementary materials that are held by third parties: existence, accessibility, and permanence. The intention is that completion of the package occurs at the time of publication when the editorial decisions are made. It also allows for better documentation of less accessible (non-public data), by treating it symmetrically from the point of view of the journal, therefore increasing the transparency of what up until now has been very opaque.

We start by providing a detailed use case. We relate our approach to existing metadata, both in terms of structure as of content, and then describe the metadata package. We conclude by discussing some usability issues for three contributors or consumers of this information, and an outlook on a possible implementation.