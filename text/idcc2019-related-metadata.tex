A number of other initiatives address the issue of reusability and replicability, some of them through proposed metadata standards.  We have endeavoured to leverage these efforts when possible (i.e., when semantics of tags overlap with our goals and when their XML schema are designed for reuse).  Our hope is that this makes both interoperability with those efforts as easy and possible, and that the use of already established and perhaps familiar tags and attributes decreases the learning curve for use of our proposed schema.  In the remainder of this section we describe related initiatives and the influence they have on our metadata design.

\subsection{DataCite}
The most related metadata vocabulary comes from \urlcite{https://www.datacite.org/}{DataCite}, which provides infrastructure to locate, identify, and cite research data. Identification is done via the DOI infrastructure for persistent identification, which has emerged as the standard for naming scholarly objects.  The DataCite metadata schema \parencite{DataCiteMetadataWorkingGroupDataCiteMetadataSchema2017a,DataCiteMetadataWorkingGroupDataCiteMetadataSchema2017} specifies elements and attributes to describe data resources for the purpose of citation, location and retrieval.  Because of the notable overlap in the purpose of DataCite  and our proposal, we make use of multiple parts of this schema. Note, however, that DataCite is targeted as describing the data products themselves, where our concern is to register the placement of those products in a repository and ancillary information about that placement.

\subsection{Re3data}
The goal of describing repositories and archives for data curation is directly addressed by the Re3data initiative \parencite{RucknagelMetadataSchemaDescription2015,Re3data.Orgre3dataorgMetadata2015}.  The goal of Re3Data is to support an online registry of research data repositories.   The mechanics underlying this is to establish a common metadata standard for describing such repositories, This metadata is then used to power a search interface.  The registry and search interface are targeted at researchers searching for the appropriate repository in which to store their data.
A primary technical output of the work of re3data is a ``Metadata Schema for Description of Research Data Repositories'' now in its 3rd version and expressed as an XML schema.  The schema addresses repository characteristics such as identification,   language, administrative contacts, subject focus, funding basis and the like.  Our work addresses repository characteristics and reuses semantics from the Re3data schema where appropriate and possible.  We will describe the details of this reuse later in this paper.

\subsection{CrossRef}
\urlcite{https://www.crossref.org/}{CrossRef}  sits functionally between our work and the two initiatives described above.  It was conceived by publishers as a DOI registry that, in addition to providing the resolution of those DOIs, stores metadata for the corresponding scholarly object.  An important aspect of this metadata are cross-references (citations) among the named objects \parencite{CrossRefRelationshipsDOIsother}.  In that sense, CrossRef acts as a ``switchboard'', documenting linkages between scholarly objects. Originally, the linkages were citations between journals, but with increasing interest in data these linkages have been expanded to include these supplementary materials.  In this context, CrossRef collaborates and interoperates with DataCite, with the former focusing on registration and description of journal articles and conference papers, and the latter on data and other supplementary artifacts .  The CrossRef schema is a relatively complex tag set for describing articles.  As our intention is to promote a lightweight approach (not necessarily exclusive but perhaps in tandem with CrossRef), we have not directly borrowed from their schema.  Also, our focus is linking to repositories or archives that contain supplementary material, as opposed to the object itself.

\subsection{Scholix}
The Scholix effort \parencite{BurtonScholixMetadataSchema2017} is also closely related to our proposed package. However, while it may lay the groundwork for the information here, it fundamentally does not have rich enough information about the linked objects to fulfill our core purpose.

\subsection{CoreTrustSeal}
Two additional related initiatives are worthy of mention.   The Core Trustworthy Data Repository Requirements \parencite{CoreTrustSealDataRepositoriesRequirements2017} are the result of work within the Research Data Alliance to establish standards for so-called ``trustworthy'' repositories.  These are repositories that meet a set of criteria that deem them dependable for the long-term curation of data.  The criteria are a mixture of technical, administrative, financial, and personnel characteristics.  The criteria are not as of yet, or planned to be, encoded in a machine-readable schema.  Instead, repositories apply for trusted status through a form that his reviewed by a human board of review.  Our proposed metadata format allows for the attribution of a repository as ``trusted'' and thus integrates minimally with the CoreTrustSeal effort.

\subsection{JATS}
The \urlcite{https://jats.nlm.nih.gov/}{JATS (Journal Article Tag Suite)}, led by the NCBI (National Center for Biotechnology Information) aims to develop specifications for standardized (XML) markup for scholarly articles.  The effort grows out of work done on so-called ``NLM DTDS'', which modelled tag sets for scholarly document structuring.  \urlcite{https://jats4r.org/}{JATS4R} (JATS for reuse) is a follow-on effort, designed to reuse and extend XML models defined by JATS, with the primary goal of facilitating reuse of existing scholarly material (publications and supplementary data). The result is a set of models specifying document structure, rather than simply metadata.  The structural elements address issues such as how to mark-up authors and affiliations, citations, data citations and the like.

\subsection{Data Accessibility Statements}
The Belmont Forum has recently started \urlcite{a project}{http://www.bfe-inf.org/resource/belmont-forum-data-publishing-policy-workshop-report-draft} to standardize a \ac{DAS}. Its goals seem to be quite similar to our project, and while independently developed, we look forward to seeing their suggestions, and will collaborate in moving that forward.