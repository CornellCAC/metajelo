We propose a metadata package that is intended to provide academic journals with a lightweight means of registering, at the time of publication, the existence and disposition of supplementary materials.  Information about the supplementary materials is, in most cases, critical for the reproducibility and replicability of scholarly results.  In many instances, these materials are curated by a third party, which may or may not follow developing standards for the identification and description of those materials.  As such, the vocabulary described here complements existing initiatives that specify vocabularies to describe the supplementary materials or the repositories and archives in which they have been deposited.  Where possible, it reuses elements of relevant other vocabularies, facilitating coexistence with them.  Furthermore, it provides an ``at publication'' record of reproducibility characteristics of a particular article that has been selected for publication.  The proposed metadata package documents the key characteristics that journals care about in the case of supplementary materials that are held by third parties: existence, accessibility, and permanence. It does so in a robust, time-invariant fashion at the time of publication, when the editorial decisions are made. It also allows for better documentation of less accessible (non-public data), by treating it symmetrically from the point of view of the journal, therefore increasing the transparency of what up until now has been very opaque.